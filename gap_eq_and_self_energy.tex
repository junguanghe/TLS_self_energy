\documentclass[]{article}

\usepackage[
top=9mm,left=10mm,bottom=6mm,right=9mm]{geometry}

\usepackage{amsmath}
\usepackage{physics}

\begin{document}

\section{Self-Energy}

The Dyson equation is
\begin{align}
    \widehat{G}^{-1} = \widehat{G}^{(0)-1} - \widehat{\Sigma}
\end{align}
we have
\begin{align}
    \widehat{G}^{(0)-1}    & = i\varepsilon_n - \xi\widehat{\tau}_3 - \Delta\widehat{\tau}_1 \\
    \widehat{\Sigma}       & \equiv \Sigma_\varepsilon + \Sigma_\Delta \widehat{\tau}_1      \\
    i\tilde{\varepsilon}_n & \equiv i\varepsilon_n - \Sigma_\varepsilon
    \rightarrow \tilde{\varepsilon}_n = \varepsilon_n + i\Sigma_\varepsilon                  \\
    \tilde{\Delta}         & \equiv \Delta + \Sigma_\Delta
\end{align}
and
\begin{align}
    \widehat{G} = (i\tilde{\varepsilon}_n - \xi\widehat{\tau}_3 - \tilde{\Delta}\widehat{\tau}_1)^{-1}
    = - \frac{i\tilde{\varepsilon}_n + \xi\widehat{\tau}_3 + \tilde{\Delta}\widehat{\tau}_1}
    {\tilde{\varepsilon}_n^2 + \xi^2 + \tilde{\Delta}^2}
\end{align}
so from $\widehat{G}^{(0)}$ to $\widehat{G}$, the change is $\Delta \rightarrow \tilde{\Delta}$
and $\varepsilon_n \rightarrow \tilde{\varepsilon}_n$.

We multiply $\widehat{G}$ from left and right to the Dyson equation, we have
\begin{align}
    1 = \widehat{G}\widehat{G}^{(0)-1} - \widehat{G}\widehat{\Sigma} \rightarrow
    1 = \widehat{\tau}_3 \widehat{G}\widehat{G}^{(0)-1} \widehat{\tau}_3
    - \widehat{\tau}_3 \widehat{G}\widehat{\Sigma} \widehat{\tau}_3 \\
    1 = \widehat{G}^{(0)-1}\widehat{G} - \widehat{\Sigma}\widehat{G} \rightarrow
    1 = \widehat{G}^{(0)-1}\widehat{\tau}_3\widehat{\tau}_3\widehat{G}
    - \widehat{\Sigma}\widehat{\tau}_3\widehat{\tau}_3\widehat{G}
\end{align}
Subtract the two equations, we have
\begin{align}
    \comm{\widehat{G}^{(0)-1}\widehat{\tau}_3-\widehat{\Sigma}\widehat{\tau}_3}{\widehat{\tau}_3\widehat{G}} & = 0 \\
    \rightarrow \comm{i\varepsilon_n\widehat{\tau}_3 + i\Delta\widehat{\tau}_2 - \widehat{\Sigma}\widehat{\tau}_3}
    {\widehat{\tau}_3\widehat{G}}                                                                            & = 0
\end{align}
In quasiclassical limit, we assume $\widehat{\Sigma}$ does not depend on $\xi$, and we can apply $\int\dd\xi$ to get the
homogeneous Eilenberger equation
\begin{align}
    \comm{i\varepsilon_n\widehat{\tau}_3 + i\Delta\widehat{\tau}_2 - \widehat{\Sigma}\widehat{\tau}_3}
    {\widehat{\mathcal{G}}} = 0
\end{align}
where the quasiclassical Green's function is
\begin{align}
    \widehat{\mathcal{G}} = \int\dd\xi\widehat{\tau}_3\widehat{G}
    = -\pi\frac{i\tilde{\varepsilon}_n\widehat{\tau}_3 + i\tilde{\Delta}\widehat{\tau}_2}
    {\sqrt{\tilde{\Delta}^2 + \tilde{\varepsilon}_n^2}}
\end{align}

The second-order self-energy is
\begin{align}
    \widehat{\Sigma}^{(2)} = \widehat{\Sigma}_{vv}^{(2)} + \widehat{\Sigma}_{vm}^{(2)}
    + \widehat{\Sigma}_{mm}^{(2)} +\widehat{\Sigma}_{nn}^{(2)}
    \equiv \widehat{\Sigma}^{(2)}_\text{elastic} + \widehat{\Sigma}^{(2)}_\text{inelastic}
\end{align}
We define the scattering rates as
\begin{align}
    \frac{\hbar}{2\pi\tau_{el}(E, T)} \equiv \hbar\left(\frac{1}{\tau_{vv}} + \frac{2N_{ge}}{\tau_{vm}}
    + \frac{1}{\tau_{mm}}\right)
\end{align}
and
\begin{align}
    \frac{\hbar}{2\pi\tau_{in}} \equiv \frac{\hbar}{\tau_{nn}}
\end{align}
The elastic part is proportional to the quasiclassical Green's function,
\begin{align}
    \widehat{\Sigma}^{(2)}_\text{elastic}(\varepsilon_n)
     & =\frac{\hbar}{2\tau_{el}}
    \frac{-i\varepsilon_n+\Delta\widehat{\tau_1}}{\sqrt{\Delta^2+\varepsilon_n^2}}
    \rightarrow \widehat{\Sigma}^{(2)}_\text{elastic}\widehat{\tau}_3 = \frac{\hbar}{2\pi\tau_{el}}\widehat{\mathcal{G}}^{(0)}
\end{align}
After renormalization, $\widehat{\mathcal{G}}^{(0)} \rightarrow \widehat{\mathcal{G}}$,
the elastic self-energy drops out of the Eilenberger equation.

For the inelastic part, if we evaluate the two Matsubara sums first, we have
\begin{align}
    \widehat{\Sigma}^{(2)}_\text{inelastic}(\varepsilon_n)
     = \frac{\hbar}{2\pi\tau_{in}}\int\dd\xi\bigg(1-N_{ge}[n_F(-\xi_\Delta)-n_F(\xi_\Delta)]\bigg)
    \frac{-i\varepsilon_n+\Delta\widehat{\tau}_1\frac{\xi_\Delta-E}{\xi_\Delta}}
    {(\xi_\Delta-E)^2+\varepsilon_n^2}
\end{align}
where we denote the transformed kinetic energy as $\xi_\Delta\equiv\text{sgn}(\xi)\sqrt{\xi^2+\Delta^2}$.
% After renormalization, we have
% \begin{align}
%     \Sigma_{\varepsilon,in}^{(2)}(\varepsilon_n) + \Sigma_{\Delta,in}^{(2)}(\varepsilon_n)\widehat{\tau}_1
%      = \frac{\hbar}{2\pi\tau_{in}}\int\dd\xi &\left(1-N_{ge}\tanh(\frac{\text{sgn}(\xi)\sqrt{\xi^2+\left(\Delta+\Sigma_{\Delta,in}^{(2)}(\varepsilon_n)\right)^2}}{2T})\right)\nonumber\\
%     & \times
%      \frac{-i\varepsilon_n+\Sigma_{\varepsilon,in}^{(2)}(\varepsilon_n)
%      +\left(\Delta+\Sigma_{\Delta,in}^{(2)}(\varepsilon_n)\right)\widehat{\tau}_1
%      \frac{\text{sgn}(\xi)\sqrt{\xi^2+\left(\Delta+\Sigma_{\Delta,in}^{(2)}(\varepsilon_n)\right)^2}-E}
%      {\text{sgn}(\xi)\sqrt{\xi^2+\left(\Delta+\Sigma_{\Delta,in}^{(2)}(\varepsilon_n)\right)^2}}}
%      {\left(\text{sgn}(\xi)\sqrt{\xi^2+\left(\Delta+\Sigma_{\Delta,in}^{(2)}(\varepsilon_n)\right)^2}-E\right)^2
%      +\left(\varepsilon_n+i\Sigma_{\varepsilon,in}^{(2)}(\varepsilon_n)\right)^2}
% \end{align}

If we evaluate the $\xi$ integral first, we have
\begin{align}
    \widehat{\Sigma}^{(2)}_\text{inelastic}(\varepsilon_n)
    = -\frac{\hbar}{2\pi\tau_{in}}\sum_{a\neq b}T^2\sum_{n_1,n_2}D_a(\varepsilon_{n_1})D_b(\varepsilon_{n_2})
    \widehat{\mathcal{G}}^{(0)}(\varepsilon_n-\varepsilon_{n_1}+\varepsilon_{n_2})\widehat{\tau}_3
\end{align}
We define the Bosonic Matsubara frequency as $\omega_m \equiv \varepsilon_{n_1}-\varepsilon_{n_2}$.
\begin{align}
    \widehat{\Sigma}^{(2)}_\text{inelastic}(\varepsilon_n)
    = -\frac{\hbar}{2\pi\tau_{in}}\sum_{a\neq b}T^2\sum_{m,n_2}D_a(\omega_m+\varepsilon_{n_2})D_b(\varepsilon_{n_2})
    \widehat{\mathcal{G}}^{(0)}(\varepsilon_n-\omega_m)\widehat{\tau}_3
\end{align}
Note that
\begin{align}
    &T\sum_{n_2}D_a(\omega_m+\varepsilon_{n_2})D_b(\varepsilon_{n_2}) \\
    = &T\sum_{n_2}\frac{1}{i\varepsilon_{n_2}+i\omega_m+\mu_f-\varepsilon_a}\times\frac{1}{i\varepsilon_{n_2}+\mu_f-\varepsilon_b}\\
    = &\frac{n_F(\varepsilon_a-i\omega_m-\mu_f)-n_F(\varepsilon_b-\mu_f)}{(\varepsilon_a-\varepsilon_b)-i\omega_m}\\
    = &\frac{n_a-n_b}{\varepsilon_{ab}-i\omega_m}
\end{align}
where we denote $n_a\equiv n_F(\varepsilon_a-\mu_f)$. Now sum over the two levels
\begin{align}
    &\sum_{a\neq b}\frac{n_a-n_b}{\varepsilon_{ab}-i\omega_m} \\
    =&\frac{N_{eg}}{E-i\omega_m} + \frac{N_{eg}}{E+i\omega_m} \\
    =&-2N_{ge}\frac{E}{E^2+\omega_m^2}
\end{align}
and we have
\begin{align}
    \widehat{\Sigma}^{(2)}_\text{inelastic}(\varepsilon_n)
    = & \frac{\hbar}{2\pi\tau_{in}}\times2N_{ge}T\sum_{m}\frac{E}{E^2+\omega_m^2}\widehat{\mathcal{G}}^{(0)}(\varepsilon_n-\omega_m)\widehat{\tau}_3\\
    \Sigma_{\varepsilon,in}^{(2)}(\varepsilon_n) + \Sigma_{\Delta,in}^{(2)}(\varepsilon_n)\widehat{\tau}_1
    = & \frac{\hbar}{\tau_{in}}N_{ge}T\sum_{m}\frac{E}{E^2+\omega_m^2}
    \frac{-i(\varepsilon_n-\omega_m) + \Delta\widehat{\tau}_1}{\sqrt{\Delta^2 + (\varepsilon_n-\omega_m)^2}}
\end{align}
After renormalization, $\widehat{\mathcal{G}}^{(0)} \rightarrow \widehat{\mathcal{G}}$,
and $\widehat{\Sigma}_\text{inelastic}^{(2)}\rightarrow \widehat{\Sigma}_\text{inelastic}^{(2,re)}$,
which includes a set of diagrams with particular `Saturn' pattern.
\begin{align}
    \widehat{\Sigma}_\text{inelastic}^{(2,re)}(\varepsilon_n)
    = & \frac{\hbar}{\pi\tau_{in}}N_{ge}T\sum_{m}\frac{E}{E^2+\omega_m^2}
    \widehat{\mathcal{G}}(\varepsilon_n-\omega_m)\widehat{\tau}_3 \\
    \Sigma_{\varepsilon,in}^{(2,re)}(\varepsilon_n) + \Sigma_{\Delta,in}^{(2,re)}(\varepsilon_n)\widehat{\tau}_1
    = & \frac{\hbar}{\tau_{in}}N_{ge}T\sum_{m}\frac{E}{E^2+\omega_m^2}
    \frac{-i(\varepsilon_n-\omega_m) + \Sigma_{\varepsilon,in}^{(2,re)}(\varepsilon_n-\omega_m)
     + \left(\Delta+\Sigma_{\Delta,in}^{(2,re)}(\varepsilon_n-\omega_m)\right)\widehat{\tau}_1}
     {\sqrt{\left(\Delta+\Sigma_{\Delta,in}^{(2,re)}(\varepsilon_n-\omega_m)\right)^2 
     + \left(\varepsilon_n-\omega_m+i\Sigma_{\varepsilon,in}^{(2,re)}(\varepsilon_n-\omega_m)\right)^2}}
\end{align}
In order to do analytic continuation, we introduce $\varepsilon_l \equiv \varepsilon_n - \omega_m$,
\begin{align}
    \widehat{\Sigma}_\text{inelastic}^{(2,re)}(\varepsilon_n)
    = & \frac{\hbar}{\pi\tau_{in}}N_{ge}T\sum_l\frac{E}{E^2+(\varepsilon_n-\varepsilon_l)^2}
    \widehat{\mathcal{G}}(\varepsilon_l)\widehat{\tau}_3 \\
    \Sigma_{\varepsilon,in}^{(2,re)}(\varepsilon_n) + \Sigma_{\Delta,in}^{(2,re)}(\varepsilon_n)\widehat{\tau}_1
    = & \frac{\hbar}{\tau_{in}}N_{ge}T\sum_l\frac{E}{E^2+(\varepsilon_n-\varepsilon_l)^2}
    \frac{-i\varepsilon_l + \Sigma_{\varepsilon,in}^{(2,re)}(\varepsilon_l)
     + \left(\Delta+\Sigma_{\Delta,in}^{(2,re)}(\varepsilon_l)\right)\widehat{\tau}_1}
     {\sqrt{\left(\Delta+\Sigma_{\Delta,in}^{(2,re)}(\varepsilon_l)\right)^2 
     + \left(\varepsilon_l+i\Sigma_{\varepsilon,in}^{(2,re)}(\varepsilon_l)\right)^2}}\label{renormalized}
\end{align}
and we substitute $i\varepsilon_n \rightarrow \varepsilon + i0^+$, we have
\begin{align}
    \widehat{\Sigma}_\text{inelastic}^{(2,re)}(\varepsilon)
    = & \frac{\hbar}{\pi\tau_{in}}N_{ge}T\sum_l\frac{E}{E^2-(\varepsilon-i\varepsilon_l)^2}
    \widehat{\mathcal{G}}(\varepsilon_l)\widehat{\tau}_3 \\
    \Sigma_{\varepsilon,in}^{(2,re)}(\varepsilon) + \Sigma_{\Delta,in}^{(2,re)}(\varepsilon)\widehat{\tau}_1
    = & \frac{\hbar}{\tau_{in}}N_{ge}T\sum_l\frac{E}{E^2-(\varepsilon-i\varepsilon_l)^2}
    \frac{-i\varepsilon_l + \Sigma_{\varepsilon,in}^{(2,re)}(\varepsilon_l)
     + \left(\Delta+\Sigma_{\Delta,in}^{(2,re)}(\varepsilon_l)\right)\widehat{\tau}_1}
     {\sqrt{\left(\Delta+\Sigma_{\Delta,in}^{(2,re)}(\varepsilon_l)\right)^2 
     + \left(\varepsilon_l+i\Sigma_{\varepsilon,in}^{(2,re)}(\varepsilon_l)\right)^2}}
\end{align}

\section{TLS induced $T_c^{tls}$ without mean-field pairing}
If we turn off the mean-field pairing, i.e. $\Delta = 0$ in Eq.~\eqref{eq:gap_eq}, then the whole system
becomes a normal metal, and the question becomes whether the interaction between electrons and TLSs
can induce a superconducting state below a critical temperature,
i.e. $\Sigma_{\Delta,in}^{(2,re)}(\varepsilon) > 0$ when $T < T_c^{tls}$,
\begin{align}
    \Sigma_{\varepsilon,in}^{(2,re)}(\varepsilon_n) + \Sigma_{\Delta,in}^{(2,re)}(\varepsilon_n)\widehat{\tau}_1
    = \frac{\hbar}{\tau_{in}}N_{ge}T\sum_l\frac{E}{E^2+(\varepsilon_n-\varepsilon_l)^2}
    \frac{-i\varepsilon_l + \Sigma_{\varepsilon,in}^{(2,re)}(\varepsilon_l)
     + \Sigma_{\Delta,in}^{(2,re)}(\varepsilon_l)\widehat{\tau}_1}
     {\sqrt{\Sigma_{\Delta,in}^{(2,re)}(\varepsilon_l)^2 
     + \left(\varepsilon_l+i\Sigma_{\varepsilon,in}^{(2,re)}(\varepsilon_l)\right)^2}}
\end{align}

When $T \geq T_c^{tls}$, we have $\Sigma_{\Delta,in}^{(2,re)} = 0$.
For the diagonal part $\Sigma_{\varepsilon,in}^{(2,re)}$, we have
\begin{align}\label{eq:diagonal_normal}
    i\Sigma_{\varepsilon,in}^{(2,re)}(\varepsilon_n)
    = \frac{\hbar}{\tau_{in}}N_{ge}T\sum_l\frac{E}{E^2+(\varepsilon_n-\varepsilon_l)^2}
    \text{sgn}\left(\varepsilon_l + i\Sigma_{\varepsilon,in}^{(2,re)}(\varepsilon_l)\right)
\end{align}
We propose an ansatz that $\Sigma_{\varepsilon,in}^{(2,re)}(\varepsilon_n)$
is an odd function of $\varepsilon_n$, and has the same sign as $\varepsilon_n$, i.e.
$\text{sgn}(\varepsilon_n) = \text{sgn}\left(i\Sigma_{\varepsilon,in}^{(2,re)}(\varepsilon_n)\right)$, and
$\Sigma_{\varepsilon,in}^{(2,re)}(\varepsilon_n) = -\Sigma_{\varepsilon,in}^{(2,re)}(-\varepsilon_n)$.
Then we have
\begin{align}
    i\Sigma_{\varepsilon,in}^{(2,re)}(\varepsilon_n)
    = \frac{\hbar}{\tau_{in}}N_{ge}T\sum_l\frac{E}{E^2+(\varepsilon_n-\varepsilon_l)^2}\text{sgn}(\varepsilon_l)
\end{align}
We seperate the positive and negative parts of $\varepsilon_l$ in the sum, and we have
\begin{align}
    i\Sigma_{\varepsilon,in}^{(2,re)}(\varepsilon_n)
    & = \frac{\hbar}{\tau_{in}}N_{ge}T\sum_{\varepsilon_l > 0}\frac{E}{E^2+(\varepsilon_n-\varepsilon_l)^2}
    - \frac{\hbar}{\tau_{in}}N_{ge}T\sum_{\varepsilon_l < 0}\frac{E}{E^2+(\varepsilon_n-\varepsilon_l)^2}
\end{align}
We define $\omega_m \equiv \varepsilon_n - \varepsilon_l = 2m\pi T$, and
\begin{align}
    i\Sigma_{\varepsilon,in}^{(2,re)}(\varepsilon_n)
    & = \frac{\hbar}{\tau_{in}}N_{ge}T\sum_{\omega_m < \varepsilon_n}\frac{E}{E^2+\omega_m^2}
    - \frac{\hbar}{\tau_{in}}N_{ge}T\sum_{\omega_m > \varepsilon_n}\frac{E}{E^2+\omega_m^2}\\
    & = \frac{\hbar}{\tau_{in}}N_{ge}T\sum_{\omega_m > -\varepsilon_n}\frac{E}{E^2+\omega_m^2}
    - \frac{\hbar}{\tau_{in}}N_{ge}T\sum_{\omega_m > \varepsilon_n}\frac{E}{E^2+\omega_m^2}\\
    & = \frac{\hbar}{\tau_{in}}N_{ge}\text{sgn}(\varepsilon_n)T
    \sum_{|\omega_m|<|\varepsilon_n|}\frac{E}{E^2+\omega_m^2} \label{eq:diagonal}
\end{align}
This solution is consistent with the ansatz.

For the off-diagonal part, when $T \rightarrow T_c^{tls}$,
we have $i\Sigma_{\varepsilon,in}^{(2,re)} \gg \Sigma_{\Delta,in}^{(2,re)} \rightarrow 0$, and
\begin{align}
    \Sigma_{\Delta,in}^{(2,re)}(\varepsilon_n)
    = \frac{\hbar}{\tau_{in}}N_{ge}T_c^{tls}\sum_l\eval{\frac{E}{E^2+(\varepsilon_n-\varepsilon_l)^2}
    \frac{\Sigma_{\Delta,in}^{(2,re)}(\varepsilon_l)}
    {\left|\varepsilon_l + i\Sigma_{\varepsilon,in}^{(2,re)}(\varepsilon_l)\right|}}_{T_c^{tls}} \label{eq:off-diagonal}
\end{align}
This is just an eigenvalue problem. For $\Sigma_{\Delta,in}^{(2,re)}(\varepsilon_n)$ to have a non-trivial solution,
we should have
\begin{align}
    \sum_l \bigg(M_{nl} - \delta_{nl}\bigg) \Sigma_{\Delta,in}^{(2,re)}(\varepsilon_l) = 0
    \leftrightarrow
    \det(M_{nl} - \delta_{nl}) = 0
\end{align}
where
\begin{align}
    M_{nl} = 
    \frac{\hbar}{\tau_{in}}N_{ge}T_c^{tls}\eval{\frac{E}{E^2+(\varepsilon_n-\varepsilon_l)^2}
    \frac{1}{\left|\varepsilon_l + i\Sigma_{\varepsilon,in}^{(2,re)}(\varepsilon_l)\right|}}_{T_c^{tls}}
\end{align}

\section{TLS induced $T_c$ shift with mean-field pairing}
With mean-field pairing, interaction with TLSs can induce a shift of $T_{c_0}$ to $T_c$.
When $T \rightarrow T_c$, we have $\Delta \ll T_c$ and $\Sigma_{\Delta,in}^{(2,re)} \ll T_c$,
the solution of the diagonal self-energy $\Sigma_{\varepsilon,in}^{(2,re)}$ is still Eq.~\eqref{eq:diagonal}.
For the off-diagonal part, the expression is similar to Eq.~\eqref{eq:off-diagonal}, which is
\begin{align}
    \frac{\Sigma_{\Delta,in}^{(2,re)}(\varepsilon_n)}{\Delta}
    = \frac{\hbar}{\tau_{in}}N_{ge}T_c\sum_l\eval{\frac{E}{E^2+(\varepsilon_n-\varepsilon_l)^2}
    \frac{1 + \Sigma_{\Delta,in}^{(2,re)}(\varepsilon_l)/\Delta}
    {\left|\varepsilon_l + i\Sigma_{\varepsilon,in}^{(2,re)}(\varepsilon_l)\right|}}_{T_c}
\end{align}
The renormalized gap equation Eq.~\eqref{eq:gap_eq_renormalized} becomes
\begin{align}
    \ln\frac{T_c}{T_{c_0}} = \pi T_c\sum_n
    \Bigg(\frac{1 + \Sigma_\Delta/\Delta}{|\varepsilon_n + i\Sigma_\varepsilon|}
    - \frac{1}{|\varepsilon_n|}\Bigg)
\end{align}


\section{Weak-Coupling Limit}
In weak-coupling limit, we assume the interaction kernel becomes
\begin{align}
    \frac{E}{E^2+(\varepsilon_n-\varepsilon_l)^2} \rightarrow
    \frac{1}{E}\Theta(E-|\varepsilon_l|)
\end{align}
then the self-energy will not depend on energy $\varepsilon_n$.
In the normal state, i.e. when $T \geq T_c$,
the equation for the diagonal self-energy is similar to
Eq.~\eqref{eq:diagonal_normal}, which is
\begin{align}
    i\Sigma_{\varepsilon,in}^{(2,re)} = \frac{\hbar}{\tau_{in}E}N_{ge}T
    \sum_{|\varepsilon_l| < E}
    \text{sgn}\left(\varepsilon_l + i\Sigma_{\varepsilon,in}^{(2,re)}\right)
\end{align}
A trivial solution is $\Sigma_{\varepsilon,in}^{(2,re)} = 0$. In this case,
when $T \rightarrow T_c$, the off-diagonal self-energy is determined by
\begin{align}\label{eq:off-diagonal_weak}
    \Sigma_{\Delta,in}^{(2,re)} = \frac{\hbar}{\tau_{in}E}N_{ge}T_c\sum_{|\varepsilon_l| < |E|}
    \eval{\frac{\Delta + \Sigma_{\Delta,in}^{(2,re)}}{|\varepsilon_l|}}_{T_c}
    = \frac{\hbar}{\tau_{in}E}N_{ge}\left(\Delta + \Sigma_{\Delta,in}^{(2,re)}\right)
    T_c\sum_{|\varepsilon_l| < |E|}\eval{\frac{1}{|\varepsilon_l|}}_{T_c}
\end{align}
\subsection{TLS induced $T_c^{tls}$}
If we once again turn off the mean-field pairing $\Delta = 0$,
the TLS induced $T_c^{tls}$ is determined by
\begin{align}
    1 = \frac{\hbar}{\tau_{in}E}N_{ge}T_c^{tls}\sum_{|\varepsilon_l| < E}
    \eval{\frac{1}{|\varepsilon_l|}}_{T_c^{tls}}
\end{align}
This is similar to the BCS gap equation, except that the cutoff is $E$ instead of
the Debye frequency $\varepsilon_D$, and the interaction strength is proportional to
$1/E$ and $N_{ge}$. Use the Digamma function in Eq.~\eqref{eq:digamma}, we have
\begin{align}
    T_c^{tls} = \frac{2e^\gamma}{\pi}E\exp\left(-\frac{\pi E}{N_{ge}\hbar/\tau_{in}}\right)
    \approx 1.13E\eval{\exp\left(-\frac{\pi E}{N_{ge}\hbar/\tau_{in}}\right)}_{T_c^{tls}}
\end{align}
Compare with the BCS gap equation near $T_c$, we have
\begin{align}
    1 = gN(0)T_c\sum_{|\varepsilon_n| < \varepsilon_D}\frac{1}{|(2n+1)\pi T_c|}
\end{align}
and the solution is
\begin{align}
    T_c = \frac{2e^\gamma}{\pi}\varepsilon_D\exp\left(-\frac{\pi}{gN(0)}\right)
    \approx 1.13\varepsilon_D\exp\left(-\frac{\pi}{gN(0)}\right)
\end{align}
\subsection{TLS induced $T_c$ shift}
In weak-coupling limit, with mean-field pairing, when $T \rightarrow T_c$, if the diagonal self-energy
$\Sigma_{\varepsilon,in}^{(2,re)} = 0$, according to the gap equation Eq.~\eqref{eq:gap_eq},
the off-diagonal self-energy $\Sigma_{\Delta,in}^{(2,re)}/\Delta = 0$ should be satisfied,
otherwise the infinite sum in the gap equation will diverge, and $T_c$ is not shifted by the interaction with TLSs.
Meanwhile, we can actually solve the off-diagonal self-energy from Eq.~\eqref{eq:off-diagonal_weak},
and the solution is not garanteed to be zero. Why will the weak-coupling limit lead to this inconsistency?

\section{Density of States}

The quasiclassical Green's function is
\begin{align}
    \widehat{\mathcal{G}} \equiv \int\dd\xi\widehat{\tau}_3\widehat{G}
    =-\pi\frac{i\tilde{\varepsilon}_n\widehat{\tau}_3 + i\tilde{\Delta}\widehat{\tau}_2}
    {\sqrt{\tilde{\Delta}^2 + \tilde{\varepsilon}_n^2}}
    =\begin{pmatrix}
         g    & f  \\
         -f^* & -g
     \end{pmatrix}
\end{align}
The density of states is
\begin{align}
    N(\varepsilon) = -\frac{1}{\pi}\Im{g(\varepsilon^R)}
    = \Im{\frac{\tilde{\varepsilon}}{\sqrt{\tilde{\Delta}^2 - (\tilde{\varepsilon} + i0^+)^2}}}
\end{align}

\section{Gap Equation}

The gap equation is
\begin{align}\label{eq:gap_eq}
    \Delta = -v_0 T\sum_{\varepsilon_n}^{\varepsilon_c}\int\frac{\dd[3]k}{(2\pi)^3}\frac{1}{2}\Tr{\widehat{G}\widehat{\tau}_1}
    \equiv g\pi T\sum_{\varepsilon_n}^{\varepsilon_c}\frac{\tilde{\Delta}}{\sqrt{\tilde{\Delta}^2 + \tilde{\varepsilon}_n^2}}
\end{align}
The digamma function is
\begin{align}\label{eq:digamma}
    K(T) \equiv \pi T \sum_{\varepsilon_n}^{\varepsilon_c}\frac{1}{|\varepsilon_n|} \approx \ln(1.13\frac{\varepsilon_c}{T})
\end{align}
Then we have
\begin{align}
    \frac{1}{g} = \pi T\sum_{\varepsilon_n}^{\varepsilon_c}
    \frac{\tilde{\Delta}/\Delta}{\sqrt{\tilde{\Delta}^2 + \tilde{\varepsilon}_n^2}}
    \overset{\text{clean limit}}{=} \pi T\sum_{\varepsilon_n}^{\varepsilon_c}
    \frac{1}{\sqrt{\Delta^2 + \varepsilon_n^2}}
    \overset{T\rightarrow T_{c_0}}{=}\pi T_{c_0}\sum_{\varepsilon_n}^{\varepsilon_c}\frac{1}{|(2n+1)\pi T_{c_0}|}
    \approx \ln(1.13\frac{\varepsilon_c}{T_{c_0}})
\end{align}
Subtract the previous two equations, we have
\begin{align}
    \ln\frac{T}{T_{c_0}} = \pi T\sum_n
    \Bigg(\frac{1 + \Sigma_\Delta/\Delta}{\sqrt{(\Delta + \Sigma_\Delta)^2 + (\varepsilon_n + i\Sigma_\varepsilon)^2}}
    - \frac{1}{|\varepsilon_n|}\Bigg)\label{eq:gap_eq_renormalized}
\end{align}

\section{The TLS model}
The Hamiltonian is
\begin{equation}
    H=
    H_{\text{BCS}}
    +
    H_\text{imp}
    +
    H_\text{e-imp}
    \,,
\end{equation}
where
\begin{equation}
    H_\text{imp}(\{\vb{X}_j\}) = \sum_{j=1}^N H_{\text{imp},j}(\vb{X}_j)
    =
    \sum_{j=1}^{N}
    \left\{
    \frac{|\vb{P}_j|^2}{2M}
    +
    U_j(\vb{X}_j)
    \right\}
    \,,
\end{equation}
$U_j$ is the potential for each impurity, which have different centers $\vb{R}_j$ and orientations $\vu{a}_j$.
$\vb{X}_j$ is the position of the $j$-th impurity.
The interaction between electrons and impurities is
\begin{equation}
    H_\text{e-imp}(\{\vb{X}_j\})
    = 
    \sum_{\alpha}\int d^3r\,
    \psi^{\dag}_{\alpha}(\vb{r})
    \sum_{j}V(\vb{r}-\vb{X}_j)
    \psi_{\alpha}(\vb{r})
    \,,
    \label{eq-e-TLS}
\end{equation}
We approximate each impurity as a TLS with local strain,
\begin{equation}
    \label{eq-H_TLS}
    H_{\text{imp}} \rightarrow
    H_\text{TLS}=\sum_j\frac{E_j}{2}\,\sigma_z(j)
    \,,
\end{equation}
Here $E_j = \sqrt{J_j^2 + \varepsilon_j^2}$.
$J_j = \ev{H_\text{imp}}{e}_j - \ev{H_\text{imp}}{g}_j$ is the tunneling matrix element.
$\varepsilon_j$ is the strain.
The interaction between electrons and TLSs is
\begin{equation}
    H_\text{e-TLS}
    =
    \sum_{\alpha}\int\dd[3]r\,\psi^\dagger_{\alpha}(\vb{r})
    \sum_j
    \bigg[
    v_j(\vb{r})\,
    + 
    m_j(\vb{r})\,\sigma_{z}(j)
    + 
    n_j(\vb{r})\,\sigma_{x}(j)
    \bigg]
    \psi_{\alpha}(\vb{r})
    \,,
\end{equation}
Here the interaction potentials $f_j(\vb{r}) = f(\vb{r}-\vb{R}_j, \vb{a}_j)$ where $f(\vb{r})$ refers to $v(\vb{r}), m(\vb{r}), n(\vb{r})$.
In momentum representation, we have $\psi^\dagger_{\alpha}(\vb{r}) = \frac{1}{\sqrt{V}}\sum_{\vb{k}'}e^{-i\vb{k}'\cdot\vb{r}}c^\dagger_{\vb{k}'\alpha}$,
\begin{equation}\label{H_TLS}
    H_\text{e-TLS}
    = \frac{1}{V}
    \sum_{\vb{k}',\vb{k};\alpha}
    c^{\dag}_{\vb{k}'\alpha}\sum_{j}
    e^{-i(\vb{k}'-\vb{k})\cdot\vb{R}_j}
    \bigg[
    v_{\vb{k}'-\vb{k}}(j)\,
    +
    m_{\vb{k}'-\vb{k}}(j)\,\sigma_{z}(j)
    +
    n_{\vb{k}'-\vb{k}}(j)\,\sigma_{x}(j)
    \bigg]\,
    c_{\vb{k}\alpha}
    \,,
\end{equation}
where $f_{\vb{k}'-\vb{k}}\equiv\int d^3r\,e^{-i(\vb{k}'-\vb{k})\cdot\vb{r}}\,f(\vb{r})$ for the three electron-TLS interactions, which have dimension of $[\text{Energy}\times\text{Volume}]$.
We use Abrikosov pseudo-fermion to factorize spin operators, $\va{\sigma}(j)=\sum_{a,b}f^{\dag}_{j,a}\,\va{\sigma}_{ab}\,f_{j,b}$
\begin{eqnarray}
    \label{eq-H_electron_pseudo_fermion}
    H_{\text{e-TLS}} 
    = \frac{1}{V}
    \sum_{\vb{k}',\vb{k};\alpha}
    c^\dagger_{\vb{k}'\alpha}\sum_{j} 
    e^{-i(\vb{k}'-\vb{k})\cdot\vb{R}_j}
    \bigg[\,v_{\vb{k}'-\vb{k}}(j)\,
    + 
    \sum_{a,b}
    \,f^{\dag}_{j,a}\,A_{\vb{k}'-\vb{k}}^{ab}(j)\,f_{j,b}\,
    \bigg]c_{\vb{k}\alpha}
\end{eqnarray}
where
\begin{align}
    A_{\vb{k}'-\vb{k}}^{ab}(j)
    =
    m_{\vb{k}'-\vb{k}}(j)\,\left(\sigma_z\right)_{ab} 
    + 
    n_{\vb{k}'-\vb{k}}(j)\,\left(\sigma_x\right)_{ab}
    =\begin{pmatrix}
        m_{\vb{k}'-\vb{k}}(j) & n_{\vb{k}'-\vb{k}}(j) \\
        n_{\vb{k}'-\vb{k}}(j) & -m_{\vb{k}'-\vb{k}}(j)
    \end{pmatrix}_{ab}
\end{align}
And we use Popov-Fedotov method to do the perturbation expansion.
The first-order terms include a v and a m term.
They are elastic and are just corrections to the chemical potential.
Note that the \textbf{unperturbed} pseudo-fermion propagator $D_{ij,ab}\equiv\delta_{ij}\delta_{ab}D_{ja}$,
\begin{align}
  \widehat{G}^{(1)}_{\vb{k}',\vb{k}} = 
  \widehat{G}^{(0)}_{\vb{k}'}
  \frac{1}{V}\sum_j
  e^{-i(\vb{k}'-\vb{k})\cdot\vb{R}_j}
  \bigg[
    v_{\vb{k}'-\vb{k}}(j) - T\sum_{n;a,b}D_{jj,ab}(\varepsilon_n)A^{ab}_{\vb{k}'-\vb{k}}(j)
  \bigg]\widehat{\tau}_3
  \widehat{G}^{(0)}_{\vb{k}}
\end{align}
Apply $\prod_{i=1}^{N}\int\frac{\dd[3]{R_i}}{V}$ to do position average, we have
\begin{align}
  \prod_{i=1}^{N}\int\frac{\dd[3]{R_i}}{V}
  \widehat{G}^{(1)}_{k'k} = 
  \widehat{G}^{(1)}_{k}\delta_{k'k} = \delta_{k'k}
  \widehat{G}^{(0)}_{k}
  \frac{1}{V}\sum_j
  \bigg[
    v_0(j) - T\sum_{n,a}D_{j,a}(\varepsilon_n)m^{aa}_0(j)
  \bigg]\widehat{\tau}_3
  \widehat{G}^{(0)}_k
\end{align}
In Maekawa's model, $m^{aa}_0(j)=0$ and $v_0(j)=v_0$.
But more generally if we do the orientation average, we also have
$\ev{m_{\vb{k}'-\vb{k}}}_\Omega=\ev{n_{\vb{k}'-\vb{k}}}_\Omega=0$.
We denote $\frac{1}{V}\sum_j\rightarrow n_\text{imp}$,
\begin{align}
    \widehat{G}^{(1)}_k = \widehat{G}^{(0)}_{k}n_\text{imp}v_0\widehat{\tau}_3\widehat{G}_k^{(0)}
\end{align}
which is exactly the static impurity result.
The second-order terms include the v-v, v-m, m-m, Saturn and the figure-eight diagrams.
\begin{align}\label{self-energy}
    \widehat{G}^{(\text{Saturn})}_{k''k}(\varepsilon_n)
    =-\frac{1}{V^2}&\sum_{ij}\sum_{abcdk'}
    e^{-i(\vb{k}''-\vb{k}')\cdot\vb{R}_i}
    e^{-i(\vb{k}'-\vb{k})\cdot\vb{R}_j}
    A_{k''-k'}^{cd}(i) A_{k'-k}^{ab}(j)\nonumber\\
    \times
    T^2&\sum_{n_1,n_2}
    D_{ij,ca}(\varepsilon_{n_1})D_{ji,bd}(\varepsilon_{n_2})
    \widehat{G}^{(0)}_{k''}\widehat{\tau}_3
    \widehat{G}^0_{k'}(\varepsilon_n+\varepsilon_{n_2}-\varepsilon_{n_1})
    \widehat{\tau}_3
    \widehat{G}^{(0)}_k.
\end{align}
In the Saturn diagram, the \textbf{unperturbed} pseudo-fermion Green's function already demands repeated scattering of the same impurity.
\begin{align}
    \widehat{G}^{(\text{Saturn})}_{k''k}(\varepsilon_n)
  =-\frac{1}{V^2}\sum_{j}e^{-i(\vb{k}''-\vb{k})\cdot\vb{R}_j}
  \sum_{abk'}
  A_{k''-k'}^{ba}(j)A_{k'-k}^{ab}(j)
  T^2\sum_{n_1,n_2}
  D_{ja}(\varepsilon_{n_1})D_{jb}(\varepsilon_{n_2})
  \widehat{G}^{(0)}_{k''}\widehat{\tau}_3
  \widehat{G}^{(0)}_{k'}(\varepsilon_n+\varepsilon_{n_2}-\varepsilon_{n_1})
  \widehat{\tau}_3
  \widehat{G}^{(0)}_k.
\end{align}
Apply $\prod_{i=1}^{N}\int\frac{\dd[3]{R_i}}{V}$ to do position average, we have
\begin{align}
    \widehat{G}^{(\text{Saturn})}_{k}(\varepsilon_n)
    = -\frac{1}{V^2}\sum_{jabk'}|A^{ab}_{k'-k}(j)|^2
    T^2\sum_{n_1,n_2}
    D_{ja}(\varepsilon_{n_1})D_{jb}(\varepsilon_{n_2})
    \widehat{G}^{(0)}_k\widehat{\tau}_3
    \widehat{G}^{(0)}_{k'}(\varepsilon_n+\varepsilon_{n_2}-\varepsilon_{n_1})
    \widehat{\tau}_3
    \widehat{G}^{(0)}_k.
\end{align}
Apply orientation average $\prod_{i=1}^{N}\int\frac{\dd\Omega_{a_j}}{4\pi}$,
which is equivalently averaging over the external momentum direction $\prod_{i=1}^{N}\int\frac{\dd\Omega_k}{4\pi}$.
Note that $\frac{1}{V}\sum_{k'}\rightarrow\frac{1}{(2\pi)^3}\int\dd[3]{k'}\approx\int\frac{\dd\Omega_{k'}}{4\pi}N(0)\int\dd\xi_{k'}$,
\begin{align}
    \widehat{G}^{(\text{Saturn})}_{k}(\varepsilon_n)
    = -\frac{1}{V}\sum_{jab}N(0)\int\dd\xi_{k'}
    \ev{|A^{ab}_{k'-k}|^2}_\Omega
    T^2\sum_{n_1,n_2}
    D_{ja}(\varepsilon_{n_1})D_{jb}(\varepsilon_{n_2})
    \widehat{G}^{(0)}_k\widehat{\tau}_3
    \widehat{G}^{(0)}_{k'}(\varepsilon_n+\varepsilon_{n_2}-\varepsilon_{n_1})
    \widehat{\tau}_3
    \widehat{G}^{(0)}_k.
\end{align}
We define the inelastic scattering time $\tau_{in}$ as $\frac{\hbar}{2\pi\tau}=n_\text{imp}N(0)\ev{|A|^2}_\Omega$,
assume the interaction potential only depends on the direction of the momentum transfer,
define quasiclassical Green's function $\widehat{\mathcal{G}}$, and assume identical TLS energy splitting,
\begin{align}
    \widehat{G}^{(\text{Saturn-inelastic})}_{k}(\varepsilon_n)
    = -\sum_{ab}
    \left(\frac{\hbar}{2\pi\tau_n}\delta^{ab} + \frac{\hbar}{2\pi\tau_m}\sigma_x^{ab}\right)
    T^2\sum_{n_1,n_2}
    D_a(\varepsilon_{n_1})D_b(\varepsilon_{n_2})
    \widehat{G}^{(0)}_k
    \widehat{\mathcal{G}}^{(0)}(\varepsilon_n+\varepsilon_{n_2}-\varepsilon_{n_1})
    \widehat{\tau}_3
    \widehat{G}^{(0)}_k.
\end{align}
Later we can still do a configuration average over the TLS energy splitting.
\end{document}